\documentclass[twoside,a4paper]{refart}
%\documentclass[pagesize,twoside,a5paper,smallborder,10pt]{refart}

% example: https://de.sharelatex.com/project/56d7250dc2b46eb00522b9c6

\usepackage[utf8]{inputenc}
\usepackage[english]{babel}
\usepackage[]{hyperref}
\usepackage{color}
\definecolor{codebg}{gray}{0.85}

\usepackage{listings} % https://de.sharelatex.com/learn/Code_listing
\lstMakeShortInline[basicstyle=\small\ttfamily\color{blue}]|

\lstset{keywordstyle=\sffamily\color{red},language=C}
\def\inliner{\lstinline[basicstyle=\ttfamily,language=R,keywordstyle={}]} % http://tex.stackexchange.com/questions/44702/can-you-change-lstinline-without-changing-the-global-lstset

% http://tex.stackexchange.com/questions/215520/output-from-tree-command-in-a-listing
\usepackage{dirtree} % http://tug.ctan.org/macros/generic/dirtree/dirtree.pdf

\lstdefinestyle{tree}{
    literate=
    {├}{{\smash{\raisebox{-1ex}{\rule{1pt}{\baselineskip}}}\raisebox{0.5ex}{\rule{1ex}{1pt}}}}1 
    {─}{{\raisebox{0.5ex}{\rule{1.5ex}{1pt}}}}1 
    {└}{{\smash{\raisebox{0.5ex}{\rule{1pt}{\dimexpr\baselineskip-1.5ex}}}\raisebox{0.5ex}{\rule{1ex}{1pt}}}}1 
}


\newcommand{\R}{R}
\def\Version#01{\def\version{0.1}}

\title{Bag to never differ---a standard for archiving research data and code with containers}
%\title{Bagtainer Specification}
\author{Daniel Nüst, o2r Team}
\date{February 2016\\ \versionnumber}

\begin{document}

\maketitle


%%%%%%%%%%%%%%%%%%%%%%%%%%%%%%%%%%%%%%%%%%%%%%%%%%%%%%%%%%%%%%%%%%%%%%%%%%%%%%%%
\section{Introduction}

Bagtainers is the working title for an implementation of the idea of executable research containers (ERC) in the context of the DFG research project Opening Reproducible Research (o2r, \url{http://o2r.info}).


%%%%%%%%%%%%%%%%%%%%%%%%%%%%%%%%%%%%%%%%%%%%%%%%%%%%%%%%%%%%%%%%%%%%%%%%%%%%%%%%
\section{Concept}

The bagtainer specification is inspired by two approaches to improve development and operation of software.

First, "convention over configuration", see \url{https://en.wikipedia.org/wiki/Convention_over_configuration}, e.g. as practiced in \href{https://books.sonatype.com/mvnref-book/reference/installation-sect-conventionConfiguration.html}{Maven}.

Second, "DevOps", see Boettiger and \url{https://en.wikipedia.org/wiki/DevOps}.

The goal is to create a directory structure with default file names and sensible defaults for settings that require only minimal configuration in 80\% of the cases, while allowing to override each of the settings if need be.
For example, the main command to compile the text manuscript in a bagtainer could be \inliner{knitr::knit("paper.*")}, with \texttt{*} being replaced by the extension of the first file named \texttt{paper}. This covers many of the potential file formats such as \texttt{Rnw} or \texttt{Rmarkdown} with the default formats. However, if a user wants to use \inliner{rmarkdown::render(..)} on a file named \texttt{publication.md}, then the default behaviour can be overwritten.

\subsection{How to interact with a bagtainer?}

The interaction steps with a bagtainer to (re-)run the contained analysis are as follows:

\begin{enumerate}
    \item (if compressed) open the bag
    \item execute the container
    \item (automatically) compare the output contained in the bag with the just created new output
\end{enumerate}


%%%%%%%%%%%%%%%%%%%%%%%%%%%%%%%%%%%%%%%%%%%%%%%%%%%%%%%%%%%%%%%%%%%%%%%%%%%%%%%%
\section{Specification}

The key words "MUST", "MUST NOT", "REQUIRED", "SHALL", "SHALL NOT", "SHOULD", "SHOULD NOT", "RECOMMENDED", "MAY", and "OPTIONAL" in this document are to be interpreted as described in \href{http://tools.ietf.org/html/rfc2119}{RFC2119}.

\subsection{Directory structure}

\dirtree{%
.1 0001.
.2 tagmanifest-md5.txt.
.2 manifest-md5.txt.
.2 bag-info.txt.
.2 bagit.txt.
.2 data.
.3 Bagtainerfile.
.3 input.
.4 Bagtainer.R.
.4 paper.Rnw.
.3 output.
.3 container.
.4 Dockerfile.
}


%%%%%%%%%%%%%%%%%%%%%%%%%%%%%%%%%%%%%%%%%%%%%%%%%%%%%%%%%%%%%%%%%%%%%%%%%%%%%%%%
\section{Examples}




%%%%%%%%%%%%%%%%%%%%%%%%%%%%%%%%%%%%%%%%%%%%%%%%%%%%%%%%%%%%%%%%%%%%%%%%%%%%%%%%
\section{Tools}

\subsection{How to use a bagtainer from \R{}}

The  R package \texttt{bagtainer} provides the following commands to build, run, and compare bagtainers.

\begin{enumerate}
    \item \colorbox{codebg}{\inliner{initialize()}} creates the directory structure of the bagtainer, including the required Dockerfile with the required dependencies
    \item 
\end{enumerate}

\subsection{How to use a bagtainer from \texttt{\$}}

We could provide bash scripts to execute containers.

\subsection{How to use a bagtainer in the cloud/the o2r web applications?}


\section{Random notes}

Execute ERC from R, node, oder from bash!
Bagordner "container", per default werden alle Dateien darin als Dockercontainer gestartet und mit dem Standardverzeichnis "input" als ÜbergabeParameter gestartet

Default Backerfile enrhält die standardKommandos die überschrieben werden können, beliebige VMs können gestartet werden

Backerfile muss so heißen und ist YAML

Must a bagtainer contain license information? It should be possible to put it in the semantic metadata

Spec could contain recommendations for bagtainers, such as adding a human readable readme with info about alle the files in the data folder

What about "Geoboxes" (similar to Belmann et al's Bioboxes) ?

self-coherent bagtainers vs. tools around the containers, data outside the containers, ...
pro "data outside containers": hashes, compatible with GitBag-light, ...
 

\section{Out of scope}

writing good documentation of the included data set/analysis/workflow


\end{document}
